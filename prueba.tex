\documentclass{article}

\usepackage[utf8]{inputenc}
\usepackage[spanish]{babel}

\title{Apuntes de programación lineal}

\author{SAlma}


\begin{document}

\maketitle
\tableofcontents
\section{introduction}

\label{sec:introduction}

La forma estandar de un problema de programación lineal es:
Dados una matriz $A$ y vectores $b,c$, maximizar $c^tx$ sujeto a
$Ax\leq b$.

Ejercicios:

1.-Un gerente esta planeando cómo distribuir la producción de dos
máquinas para ser manufacturado cada productorequiere cierto tiempo(en
horas) en cada una de las maquinas

El tiempo reuqerido esta resumido en la siguiente tabla:

\begin{tabular}{|c|c|c|}
  \hline
  &A&B\\
  \hline
  Maquina 1&1&2\\
  \hline
  Maquina2&1&1\\
  \hline      
\end{tabular}


\end{document}
